\section{Introduction}

\ednote{Make Introduction}
\subsection{Related Work}
Standard related work stuff here
\ednote{Start Related Work here}
Especially in recent years, the development and widespread use of devices to record and process motion data, such as smartphones with cameras, cars with automatic GPS tracking, or public CCTV cameras, has lead to a need for techniques for motion analysis, recognition and synthesis.

These techniques have applications in fields as diverse as entertainment \ednote{should I add examples for each? I can think of some}, industry, healthcare, and research. This diversity and the sheer quantity of data has necessitated that techniques are efficient, unsupervised, and produce high quality, accurate results.

One of the desired capabilities is \ednote{One ofthe things we want to do, but phrased better, I think this is awkward} is automatic motion segmentation. It is interesting for two main reasons. For one, being able to tell when a motion changes is interesting in and of itself, imagine a better on a horse race wanting to know when a horse starts galloping versus cantering. The other reason is that automatic motion segmentation is is crucial for developing training data for other techniques such as motion recognition. The amount of samples required alone makes costly manual segmentation unfeasible, and additionally we would like accurate, standardized models of training data.

What are the properties that our automatic segmentation should have? 



\ednote{Make Transition nicer}
SFA is nice, for reasons (universality), and it is also an unsupervised technique to get some classification data from a timeseries. Thus we want to try to use it for segmentation of timeseries data as well.